%----------------------------------------------------------------------------------------
%	PACKAGES AND OTHER DOCUMENT CONFIGURATIONS
%----------------------------------------------------------------------------------------

\documentclass[
10pt, % Main document font size
a4paper, % Paper type, use 'letterpaper' for US Letter paper
oneside, % One page layout (no page indentation)
%twoside, % Two page layout (page indentation for binding and different headers)
headinclude,footinclude, % Extra spacing for the header and footer
BCOR5mm, % Binding correction
]{scrartcl}

\input{structure.tex} % Include the structure.tex file which specified the document structure and layout

\hyphenation{Fortran hy-phen-ation} % Specify custom hyphenation points in words with dashes where you would like hyphenation to occur, or alternatively, don't put any dashes in a word to stop hyphenation altogether

%----------------------------------------------------------------------------------------
%	TITLE AND AUTHOR(S)
%----------------------------------------------------------------------------------------

\title{\normalfont\spacedallcaps{ECON 281 Notes}}

%\subtitle{Subtitle} % Uncomment to display a subtitle

\author{\spacedlowsmallcaps{Connor H. McLaughlan}}

\date{} % An optional date to appear under the author(s)

%----------------------------------------------------------------------------------------

\begin{document}

%----------------------------------------------------------------------------------------
%	HEADERS
%----------------------------------------------------------------------------------------

\renewcommand{\sectionmark}[1]{\markright{\spacedlowsmallcaps{#1}}} % The header for all pages (oneside) or for even pages (twoside)
%\renewcommand{\subsectionmark}[1]{\markright{\thesubsection~#1}} % Uncomment when using the twoside option - this modifies the header on odd pages
\lehead{\mbox{\llap{\small\thepage\kern1em\color{halfgray} \vline}\color{halfgray}\hspace{0.5em}\rightmark\hfil}} % The header style

\pagestyle{scrheadings} % Enable the headers specified in this block

%----------------------------------------------------------------------------------------
%	TABLE OF CONTENTS & LISTS OF FIGURES AND TABLES
%----------------------------------------------------------------------------------------

\maketitle % Print the title/author/date block

\setcounter{tocdepth}{2} % Set the depth of the table of contents to show sections and subsections only

\tableofcontents % Print the table of contents

\listoffigures % Print the list of figures

\listoftables % Print the list of tables

%--------------------------------------------------
%	ABSTRACT
%--------------------------------------------------

\section*{Abstract} % This section will not appear in the table of contents due to the star (\section*)

%--------------------------------------------------
%	AUTHOR AFFILIATIONS
%--------------------------------------------------

\let\thefootnote\relax\footnotetext{* \textit{BSc. Computer Science, University of Alberta}}

%--------------------------------------------------

\newpage % Start the article content on the second page, remove this if you have a longer abstract that goes onto the second page

%--------------------------------------------------
%	INTRODUCTION
%--------------------------------------------------

\section*{Introduction}

Notes for ECON 281, intermediate microeconomics.

\newpage
%--------------------------------------------------
%	Chapter 1
%--------------------------------------------------
\section{Chapter}

TODO


\newpage
%--------------------------------------------------
%	Chapter 2
%--------------------------------------------------
\section{Chapter}

TODO


\newpage
%--------------------------------------------------
%	Chapter 3
%--------------------------------------------------
\section{Chapter}

TODO



%--------------------------------------------------
%	Chapter 4: Optimal Choice
%--------------------------------------------------
\section{Consumer Choice}

\subsection{The Budget Constraint}

\newtheorem*{bconstraint}{Budget Constraint}
\begin{bconstraint}
    defines the set of baskets that a consumer can purchase with a
    limited amount of income. Expressed as

    \[P_xx + P_yy \le I\]
\end{bconstraint}

\newtheorem*{bline}{Budget Line}
\begin{bline}
    indicates all the combinations of one good $x$ and another $y$ that a
    particular consumer can purchase if he spends all his income on the 
    two goods. Expressed as 

    \[P_xx + P_yy = I\]
\end{bline}


\begin{figure}[tb]
\centering 
\includegraphics[width=0.75\columnwidth]{budgetLine} 
\caption[Budget Line]{Budget line.}
\label{fig:budgetLine} 
\end{figure}

Figure ~\ref{fig:budgetLine} shows the graph of a budget line in units
of a particular good, and how the 
most optimal choice is where all income is spent (result is on the line).
If all money is spent on movies, 8 units can be afforded. This is found 
by $\frac{I}{P_y} = 8$.

The slope of the budget line shows how many units of one good
have to be given up to get another good.

\[\frac{\Delta y}{ \Delta x} = \frac{-P_x}{ P_y}\]

\par

An increase in income does not affect the slope because all goods can 
be purchased in excess of the amount that was afforded by the 
previous income.\par

A change in price does affect the slope.\par

%=====================================
\subsection{Optimal Choice}

\newtheorem*{oChoice}{Optimal Choice}
\begin{oChoice}
    Consumer choice of a basket of goods that (1) maximizes 
    satisfaction (utility) while (2) allowing him to live within his 
    budget constraint.
\end{oChoice}

\newtheorem*{intOptimum}{Interior Optimum}
\begin{intOptimum}
    An optimal basket at which a consumer will be purchasing positive
    amounts of all commodities.
\end{intOptimum}

\newtheorem*{expMin}{Expenditure Minimization Problem}
\begin{expMin}
    Consumer choice between goods that will minimize 
    total spending while achieving a given level of utility.
\end{expMin}

To keep things simple, time is not a factor of the budget line.\par

Let $U(x,y)$ represent the \textif{consumer's utility from purchasing $x$ units of good one, and $y$ units of good two.} \par

Let max $U(x,Y)$ represent the values $(x,y)$ that give the maximum 
utility. \par



\begin{figure}[tb]
\centering 
\includegraphics[width=0.75\columnwidth]{indCurve} 
\caption[Indifference Curve]{Indifference Map}
\label{fig:indifferenceCurve} 
\end{figure}


Figure ~\ref{fig:indifferenceCurve} shows the graph of 
indifference curves, where the further out a curve is, the 
higher the utility.\par

At the optimal basket A the budget line is tangent to an 
indifference curve. This is the curve with the highest attainable 
utility. Any other basket would be unobtainable because it would 
be higher than the income of the buyer.\par

A slight movement along the budget line results in a change 
in the utility. This is because the indifference curves 
slope inwards towards the origin.\par

The slope of the indifference curve is 

\[\frac{-MU_x}{MU_y} = -MRS_(x,y)\]

where MU is the marginal utility, and MRS is the marginal rate 
of substitution. This equation can be rewritten as 

\[\frac{-MU_x}{P_x} = -MRS_(MU_y,P_y)\]

which shows that an interior optimum is found when any surplus dollar spent on one good gives the same utility if it were spent on the other good.\par

\begin{figure}[tb]
\centering 
\includegraphics[width=0.75\columnwidth]{expMin} 
\caption[Expenditure Minimization]{Expenditure Minimization}
\label{fig:expMin} 
\end{figure}

In figure ~\ref{fig:indifferenceCurve} we are given an 
indifference map comprised of utility functions 
given by $U(x,y) = xy$. As learned in the previous chapter, 
$MU_x = y$, and $MU_y = x$. Point $B$ in the graph would have
$MU_x = B_2$ and $MU_y = A_2$. It is clear that this is not equal to 
the slope of the indifference curve, therefore basket $B$ is not 
optimal.\par

 
%=====================================
\subsection{Consumer Choice with Composite Goods}

\newtheorem*{compGood}{Composite Good}
\begin{compGood}
    A good that represents the collective expenditures on every 
    other good except the commodity being considered.
\end{compGood}

It is often useful to plot a specific good on the horizontal axis, 
and every other good in the market on the vertical axis. 
The good on the vertical axis is called a composite good because it is 
made up of many goods. By convention, the price of a composite good is 
$P_y = 1$. The total expenditure on the compisite good is therefore 
$P_yy = y$.

\begin{figure}[tb]
\centering 
\includegraphics[width=0.75\columnwidth]{compGood} 
\caption[Composite Good]{composite good}
\label{fig:compGood} 
\end{figure}

\[P_y = 1\]
\[\frac{-P_h}{P_y} = -P_h\]



%=====================================
\subsection{Revealed Preference}

TODO


\newpage
%--------------------------------------------------
%	Chapter 5
%--------------------------------------------------
\section{The Theory of Demand}


%=====================================
\subsection{Optimal Choice and Demand}

\newtheorem*{pcc}{Price Consumption Curve}
\begin{pcc}
    The set of utility-maximizing baskets as the price of one good 
    varies (holding constant the prices of other goods and income.
\end{pcc}

\newtheorem*{icc}{Income Consumption Curve}
\begin{icc}
    The set of utility-maximizing baskets as income varies (and prices
    are held constant
\end{icc}

\newtheorem*{engel}{Engel curve}
\begin{engel}
    A curve that relates the amount of a commodity purchased to income,
    holding constant the prices of all goods.
\end{engel}

\newtheorem*{normalGood}{Normal Good}
\begin{normalGood}
    A good that a consumer purchases more of as income increases. 
    Elasticity is positive.
\end{normalGood}

\newtheorem*{infGood}{Inferior Good}
\begin{infGood}
    A good that consumers purchase less of as income increases. 
    Elasticity is negative.
\end{infGood}

\begin{figure}[tb]
\centering 
\includegraphics[width=0.75\columnwidth]{pcc} 
\caption[Price Consumption Curve]{Price Consumption Curve}
\label{fig:pcc} 
\end{figure}

In figure ~\ref{fig:pcc} notice that as the price of biscuits falls, 
more biscuits can be purchased, resulting in a higher utility.


\begin{figure}[tb]
\centering 
\includegraphics[width=0.75\columnwidth]{icc} 
\caption[icc]{Income Consumption Curve}
\label{fig:icc} 
\end{figure}

In figure ~\ref{fig:icc} it can be seen how as an individual's income
changes, the budget line is redrawn (with the same slope) in the 
direction of the shift in income.\par

Another way to show how an individual's choice of a particular good 
changes with income is to draw an Engel curve, a graph relating the 
amount of food consumed to the level of income. Food is shown 
(in units consumed) on the x-axis, income on the y-axis.


%=====================================
\subsection{Change in the Price of a Good: Substitution Effect and Income Effect}

\newtheorem*{subEff}{Substitution Effect}
\begin{subEff}
    The change in the amount of a good that would be consumed 
    as the price of that good changes, holding constant all 
    other prices and the level of utility
\end{subEff}

\newtheorem*{incEff}{Income Effect}
\begin{incEff}
    The change in the amount of a good that a consumer would buy 
    as purchasing power changes, holding all prices donstant.
\end{incEff}

\newtheorem*{griffenGood}{Griffen good}
\begin{griffenGood}
    A good so strongly inferior that the income effect outweighs 
    the substitution effect, resulting in an upward-sloping 
    demand curve over some region of prices.
\end{griffenGood}


%=====================================
\subsection{Change in the Price of a Good: The Concept of Consumer Surplus}

\newtheorem*{consumerSur}{Consumer Surplus}
\begin{consumerSur}
    The difference between the maximum amount a consumer is willing 
    to pay for a good and the amount he or she must actually 
    pay when purchasing it.
\end{consumerSur}

\newtheorem*{compensatingVariation}{Compensating Variation}
\begin{compensatingVariation}
    A measure of how much money a consumer would be willing to give up
    after a reduction in the price of a good to be just as well off
    as before the price decrease
\end{compensatingVariation}

\newtheorem*{equivalentVariation}{Equivalent Variation}
\begin{equivalentVariation}
    A measure of how much additional money a consumer would need 
    before a price reduction to be as well off as after the price 
    decrease.
\end{equivalentVariation}


%=====================================
\subsection{Market Demand}

\newtheorem*{networkExternalities}{Network Externalities}
\begin{networkExternalities}
    A demand characteristic present when the amount of a good demanded
    by one consumer depends on the number of other consumers who 
    purchase the good.
\end{networkExternalities}

\newtheorem*{bandwagonEffect}{Bandwagon Effect}
\begin{bandwagonEffect}
    A positive network externality that refers to the increase in each
    consumer's demand for a good as more consumer buy the good.
\end{bandwagonEffect}

\newtheorem*{snobEffect}{Bandwagon Effect}
\begin{snobEffect}
    A negative network externality that refers to the decrease 
    in each consumer's demand as more consumers buy the good.
\end{snobEffect}


%=====================================
\subsection{The Choice of Labor and Leisure}

\subsubsection{As wages rise, leisure first decreases, then increases}

Leisure refers to all nonwork activities. The consumer's utility $U$ 
depends on the amount of leisure time and number of units of a composite 
good he can buy. The consumer's decision can be represented on 
the optimal choice diagram, with leisure on the x-axis, and 
composite goods and income on the y-axis.\par


\subsubsection{Backward-bending Supply of Labor}

\begin{figure}[tb]
\centering 
\includegraphics[width=0.75\columnwidth]{laborSupply} 
\caption[laborSupply]{Labor Supply Curve}
\label{fig:laborSupply} 
\end{figure}

Higher wages decreases the supply of labor, this is because people 
don't need to work as much to earn enough for leisure. This results in 
a backward bending labor supply curve.\par

This induces the consumer to substitute composite goods for leisure, 
resulting in an increase of labor supplied.\par


%=====================================
\subsection{Consumer Price Indices}

The ideal CPI (Consumer Price Index) would be a ratio between 
this year's expenses to last year's expenses. 
\begin{example}
    If this year's expenses are \$720, and last years were \$480, 
    the ideal CPI calculation would be $\frac{\$720}{\$480} = 1.5$
    This implies that the cost of living this year is 50\% more 
    expensive than last year.\par

    Historically the government has tracked the price of a fixed market 
    basket of goods in order to calculate CPIs.


\newpage
%--------------------------------------------------
%	Chapter 6
%--------------------------------------------------
\section{Inputs and Production Functions}

\subsection{}
\subsection{}
\subsection{}

\newtheorem*{prodFun}{Production Function}
\begin{prodFun}
    Function that shows the maximum quantity of output a firm can 
    produce given a quantity of inputs.
    \[Q = f(L,K)\]
\end{prodFun}

\newtheorem*{prodSet}{Production Set}
\begin{prodSet}
    Set of feasible combinations of inputs and outputs.
\end{prodSet}


\newtheorem*{techIneff}{Technically Inefficient}
\begin{techIneff}
    Set of points in the production set at which the firm is producing 
    less from it's labor than it could.
\end{techIneff}

\newtheorem*{techEff}{Technically Efficient}
\begin{techEff}
    Set of points in the production set where the firm is producing 
    as much output as possible given the inputs.
\end{techEff}

\newtheorem*{laborReq}{Labor Requirements Function}
\begin{laborReq}
    A function that indicates minimum labor required to produce a 
    given amount of output.
    \[L = g(Q)\]
    For example, if $Q = \sqrt{L}$ then $L = Q^2$.
\end{laborReq}

\newtheorem*{margLab}{Increasing Marginal Returns to Labor}
\begin{margLab}
    Region along the total product function where output rises with 
    additional labor at an increasing rate.
\end{margLab}
\newtheorem*{dimMargLab}{Diminishing Marginal Returns to Labor}
\begin{dimMargLab}
    Region on the curve where output increases with additional labor, 
    but at a decreasing rate.
\end{dimMargLab}

\newtheorem*{dimTotLab}{Diminishing Total Returns to Labor}
\begin{dimTotLab}
    Region on the total product function where an increase in 
    labor decreases output.
\end{dimTotLab}

\newtheorem*{apLab}{Average Product of Labor}
\begin{apLab}
    The average amount of output per unit of labor.
    \[AP_L = \frac{total\ product}{quantity\ of\ labor} = \frac{Q}{L}\]
\end{apLab}

\newtheorem*{mpLab}{Average Product of Labor}
\begin{mpLab}
    Rate at which total output changes as quantity of labor 
    used is changed.
    \[MP_L = \frac{change\ in\ total\ product}{change\ in\ quantity\ of\ labor} = \frac{\Delta Q}{\Delta L}\]
\end{mpLab}

\newtheorem*{lawDimMarg}{Law of Diminishing Marginal Returns}
\begin{lawDimMarg}
    As the usage of one input increases, the quantities of other 
    inputs being held fixed, a point will be reached beyond which 
    the marginal product of the variable input will decrease.
\end{lawDimMarg}

\newtheorem*{totProdHill}{Total Product Hill}
\begin{totProdHill}
    A three-dimensional graph of a production function.
\end{totProdHill}

\newtheorem*{isoquant}{Isoquant}
\begin{isoquant}
    A curve that shows all combinations of labor (L) and capital (K) 
    that can produce a given level of output.
\end{isoquant}

\newtheorem{uneconReg}{Uneconomic Region of Production}
\begin{uneconReg}
    Region of upward-sloping/backward-bending isoquants. At least one 
    input has a negative marginal product.
\end{uneconReg}

\newtheorem{econReg}{Economic Region of Production}
\begin{econReg}
    The region where the isoquants are downward sloping.
\end{econReg}

\newtheorem*{margTechSub}{Marginal Rate of Technical Substitution of Labor for Capital}
\begin{margTechSub}
    Rate at which the quantity of capital can be reduced for every 
    one-unit increase in the quantity of labor, holding output constant.
    $MRTS_(L,K)$ is a measure of the negative slope of an isoquant.
\end{margTechSub}

\newtheorem*{dimMagTech}{Diminishing Marginal Rate of Technical Substitution}
\begin{dimMagTech}
    The marginal rate of technical substitution of labor for capital 
    diminishes as the quantity of labor increases along an isoquant 
    (feature of a production function).
    \[\Delta Q = change\ in\ output\]
    \[change\ in\ output\ from\ change\ in\ quantity\ of\ capital = (\Delta K)(MP_K)\]
    \[change\ in\ output\ from\ change\ in\ quantity\ of\ labor = (\Delta L)(MP_L)\]
    \[\Delta Q = change\ in\ output\ from\ change\ in\ quantity\ of\ capital\]\[+ change\ in\ output\ from\ change\ in\ quantity\ of\ labor\]
    \[\frac{-\Delta K}{\Delta L} = \frac{MP_L}{MP_K}\]
    This shows that the marginal rate of technical substitution of labor for capital
    ($MRTS_{L,K}$)is equal to the ratio of the marginal product of labor ($MP_L$) to 
    the marginal product of capital ($MP_K$).
\end{dimMagTech}


%----------------------
\subsection{Substitutability Amog Inputs}

Will cover the ease or difficulty with which a firm can substitute between 
different inputs of production. \par

\begin{figure}[tb]
\centering 
\includegraphics[width=0.95\columnwidth]{inputOpportunities} 
\caption[]{Input Substitution Opportunities}
\label{fig:inputOpportunities} 
\end{figure}

These two production functions (Figure ~\ref{fig:inputOpportunities}/6.11) differ in the ease with which the firms 
can substitute labor ($L$) and capital ($K$).\par

The first curve (a) has limited opportunity for substitution because any increase in 
one input would yield a small deacrease in the other.\par

In contrast, the second curve (b) has a more linear slope, which results in a 
far greater rate of substitution.\par

A right movement along the curve (a) results in $MRTS_{L,K}$ becoming almost zero 
(slope is almost zero), while a left movement along the curve results in the 
$MRTS_{L,K}$ becoming infinite (negative of the slope is very steep).\par

\textbf{This suggests the ease or difficulty with which a firm may substitute inputs is 
directly related to the derivative (slope) of the isoquant.}\par

$MRTS_{L,K}$ changes \textbf{substantially} along the isoquant if the production
function offers limited input substitution opportunitites.\par

$MRTS_{L,K}$ changes \textbf{gradually} along the isoquant if the production
function offers substantial input substitution opportunities.\par


%-----------------------------------------------------
\subsubsection{Elasticity of Substitution}

\newtheorem*{elastSub}{elasticity of substitution}
\begin{elastSub}
    Measure of how easily a firm may substitute labor for capital.
    {\bf Often represented by \sigma .}
    Typically any value greater than, or equal to, zero.
    If \sigma is close to zero, opportunity to substitute inputs is low.
    If \sigma is high
\end{elastSub}

\begin{figure}[tb]
\centering 
\includegraphics[width=0.95\columnwidth]{elastSub} 
\caption[]{Elasticity of Substitution of Labor for Capital}
\label{fig:elastSub} 
\end{figure}

Figure ~\ref{fig:elastSub}: Elasticity of substitution describes the firm's input substitution opportunities. 
It describes how quickly the $MRTS_{L,K}$ (marginal rate technical substitution) 
changes along the isoquant curve (derivative of $MRTS_{L,K}$).\par

As labor is substituted for capital, the ratio of the {\bf quantity of capital} to the 
{\bf quantity of labor}, $K/L$, must fall. This ratio is known as the 
{\bf capital-labor ratio}. 
{\bf $\sigma$ measures the percent change in the capital-labor ratio for each 1\% change in $MRTS_{L,K}$ (as we move along the isoquant).}
\par

\begin{equation}
    \begin{aligned}
        \sigma &= \frac{\%\ change\ in\ capital\ labor\ ratio}{\%\ change\ in\ MRTS_{L,K}}\\
               &= \frac{\% \Delta \frac{K}{L}}{\% \Delta MRTS_{L,K}}\\
    \end{aligned}
\end{equation}

Figure ~\ref{fig:elastSub} illustrates the elasticity of substitution. 
Given the production function $f(L,K)$, as a firm moves from $f(5,20)$ to $f(10,10)$, the capital-labor ratio changes from $\frac{20}{5} = 4$ to $\frac{10}{10} = 1$ (-75\%), as does the $MRTS_{L,K}$. Thus,

\begin{equation*}
    \begin{aligned}
        MRTS_{L,K} &= -\frac{\Delta K}{\Delta L}\\
        MRTS_{L,K}^A &= -\frac{20}{5} = -4\\
        MRTS_{L,K}^B &= -\frac{10}{10} = -1\\
    \end{aligned}
\end{equation*}
\begin{equation*}
    \begin{aligned}
        \sigma &= \frac{\% \Delta \frac{K}{L}}{\% \Delta MRTS_{L,K}}\\
               &= \frac{100 ( \frac{4-1}{4} ) }{ 100(\frac{(-4)-(-1)}{-4})}\\
               &= \frac{\frac{4-1}{4}}{\frac{4-1}{4}}\\
               &= 1
    \end{aligned}
\end{equation*}

the elasticity of substitution over this interval is 1 ($-75\%/-75\% = 1$).
\par

\begin{tcolorbox}
    \textbf{Calculating the Elasticity of Substitution from a Production Function}\\
    \textbf{Problem}\\
    Consider a production function $Q = \sqrt{KL}$, with marginal products 
    $MP_L = \frac{1}{2} \sqrt{\frac{K}{L}}$ and
    $MP_K = \frac{1}{2} \sqrt{\frac{L}{K}}$.
    \par

    Show that the elasticity of substitution for this production function is equal to 1 no matter the values of $K$ and $L$.
    \par

    \textbf{Solution}\\
    First, note that
    \[MRTS_{L,K} = \frac{MP_L}{MP_K}\]
    This implies
    
    \begin{equation*}
        \begin{aligned}
            MRTS_{L,K} &= \frac{\frac{1}{2} \sqrt{\frac{K}{L}} }{ \frac{1}{2} \sqrt{\frac{L}{K}}}\\
                       &= \frac{K}{L}\\
        \end{aligned}
    \end{equation*}

    Since $MRTS_{L,K} = \frac{K}{L}$, it follows that $\%MRTS_{L,K}$ will be exactly equal to $\%\Delta \frac{K}{L}$.
    \par

    In other words, since the marginal rate of substitution of labor for capital equals the capital-labor ratio, the percent change of both is also equal.
    \par

    Using the definition of the elasticity of substitution,
    \begin{equation*}
        \begin{aligned}
            \sigma &= \frac{\% \Delta (\frac{K}{L})}{\% \Delta (MRTS_{L,K})}\\
                   &= \frac{\% \Delta (\frac{K}{L})}{\% \Delta (\frac{K}{L})}\\
                   &= 1
        \end{aligned}
    \end{equation*}
\end{tcolorbox}


\begin{itemize}
    \item Typically any value greater than, or equal to, zero.

    \item If $\sigma$ (elasticity of substitution) is close to zero, opportunity to substitute inputs is low.

    \item If $\sigma$ is large, there is substantial opportunity to substitute inputs. 
        This corresponds to the fact that if $\sigma$ is large, $\%\Delta MRTS_{L,K}$ (percent change $MRTS_{L,K}$) is small.
\end{itemize}


%-----------------------------------------------------
\subsubsection{Special Production Functions}

The relationship between 
\begin{itemize}
    \item curvature of isoquants
    \item input stability
    \item the elasticity of substitution
\end{itemize}
is most apparent when contrasting a number of special production functions.
\par

Four frequently used special production functions will be compared:
\begin{enumerate}
    \item linear production function
    \item fixed-proportions production function
    \item Cobb-Douglas production function
    \item constant elasticity of substitution production function
\end{enumerate}
\par


\paragraph{Linear Production Function (Perfect Substitutes)}

\begin{figure}[tb]
\centering 
\includegraphics[width=0.95\columnwidth]{linearProdFunc} 
\caption[]{Linear Production Function}
\label{fig:linearProdFunc} 
\end{figure}

A production function of the form
\[Q = aL + bK\]
where $a$ and $b$ are constants.
\par

\textbf{Example:} A manufacturing process may require fuel for input. 
Maybe a given amount of natural gas can be substituted for a quantity of kerosene.
This is an example of a linear production function.
In this case, the MRTS of natural gas for kerosene is constant.
\par

Let's say a firm needs to store 200GB of data. 
It can use two types of hard drives for this purpose: low-capacity or high-capacity.
High-capacity drives can store 20GB of data, low-capacity can store 10GB.
The production function would be:
\[Q = 20H + 10L\]
where Q would be the total GB the firm can store.
\par

\textbf{A linear production function has linear isoquants.}
\par

Linear production functions are said to have \textbf{perfect substitutes}.
\par


\paragraph{Fixed-Proportions Production Function (Perfect Complements)}

\begin{figure}[tb]
\centering 
\includegraphics[width=0.95\columnwidth]{fixedPropProdFunc} 
\caption[]{Fixed-Proportions Production Function}
\label{fig:fixedPropProdFunc}
\end{figure}

A production function in which the inputs must be combined in fixed proportions.
The inputs for this production function are called \textbf{perfect complements}.
\par

\textbf{Example:} Hydrogen and oxygen are perfect complements because adding more hydrogen to the same amount of oxygen does not result in more $H_2O$. Thus, the production function becomes
\[Q = \min (\frac{H}{2}, O )\]
\par

Fixed-proportion prod. funcs. have an \textbf{elasticity of substitution equal to zero }($\sigma = 0$).
This is because no amount of one input can be substituted for another (think about $H_2O$).


\paragraph{Cobb-Douglas Production Function}

\begin{figure}[tb]
\centering 
\includegraphics[width=0.95\columnwidth]{cobbProdFunc} 
\caption[]{Cobb-Douglas Production Function}
\label{fig:cobbProdFunc}
\end{figure}

Intermediate between linear prod-func and fixed proportions prod-func. 
Curve is given by:
\[Q = AL^{\alpha}K^{\beta}\]
where $A$, $\alpha$, and $\beta$ are positive constants (and $L$ = labor, $K$ = capital).
Capital and labor can be substituted for each other. 
Unlike a linear production function, the rate at which labour can be substituted for capital is not constant along the isoquant.


\newpage
%--------------------------------------------------
%	Chapter 7
%--------------------------------------------------
\section{Chapter}

7


\newpage
%--------------------------------------------------
%	Chapter 8
%--------------------------------------------------
\section{Chapter}

8


\newpage
%--------------------------------------------------
%	Chapter 9
%--------------------------------------------------
\section{Chapter}

9


\newpage
%--------------------------------------------------
%	Chapter 10
%--------------------------------------------------
\section{Chapter}

10


\newpage
%--------------------------------------------------
%	Chapter 11
%--------------------------------------------------
\section{Chapter}

11


\newpage
%--------------------------------------------------
%	Chapter 12
%--------------------------------------------------
\section{Chapter}

12





\end{document}
